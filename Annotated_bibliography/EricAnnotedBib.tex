\documentclass{sig-alternate}

\newcommand{\mycomment}[1]{{\bf #1}}
%\newcommand{\mycomment}[1]{}

\begin{document}
%
% --- Author Metadata here ---
\conferenceinfo{UMM CSci Senior Seminar Conference}{Morris, MN}
%\CopyrightYear{2007} % Allows default copyright year (200X) to be over-ridden - IF NEED BE.
%\crdata{0-12345-67-8/90/01}  % Allows default copyright data (0-89791-88-6/97/05) to be over-ridden - IF NEED BE.==
% --- End of Author Metadata ---

\title{
Evolving Programs at Bytecode Level
}
%
% You need the command \numberofauthors to handle the 'placement
% and alignment' of the authors beneath the title.
%
% For aesthetic reasons, we recommend 'three authors at a time'
% i.e. three 'name/affiliation blocks' be placed beneath the title.
%
% NOTE: You are NOT restricted in how many 'rows' of
% "name/affiliations" may appear. We just ask that you restrict
% the number of 'columns' to three.
%
% Because of the available 'opening page real-estate'
% we ask you to refrain from putting more than six authors
% (two rows with three columns) beneath the article title.
% More than six makes the first-page appear very cluttered indeed.
%
% Use the \alignauthor commands to handle the names
% and affiliations for an 'aesthetic maximum' of six authors.
% Add names, affiliations, addresses for
% the seventh etc. author(s) as the argument for the
% \additionalauthors command.
% These 'additional authors' will be output/set for you
% without further effort on your part as the last section in
% the body of your article BEFORE References or any Appendices.

\numberofauthors{1} %  in this sample file, there are a *total*
% of EIGHT authors. SIX appear on the 'first-page' (for formatting
% reasons) and the remaining two appear in the \additionalauthors section.
%
\author{
% You can go ahead and credit any number of authors here,
% e.g. one 'row of three' or two rows (consisting of one row of three
% and a second row of one, two or three).
%
% The command \alignauthor (no curly braces needed) should
% precede each author name, affiliation/snail-mail address and
% e-mail address. Additionally, tag each line of
% affiliation/address with \affaddr, and tag the
% e-mail address with \email.
%
% 1st. author
\alignauthor
Eric C. Collom
}

\maketitle
\begin{abstract}
This paper discusses using x86 assembly language and Java Bytecode to evolve programs. 
\end{abstract}


\category{Software Engineering}
\mycomment{I will change category later}

\terms{}
\mycomment{I will change general terms later}

\keywords{ACM proceedings, \LaTeX, text tagging}

\mycomment{I will change the keywords later}

\section{Introduction}
I plan to focus on the topic of taking programs that aren't designed to be evolved and evolving them through bytecode.  

In plan to use the following sources:~\cite{Assembly:2010} and ~\cite{FINCH:2010} as my main
sources. I will possibly use ~\cite{GISMOE:2012} and ~\cite{UCL:2012} as more information on evolving programs for different EC purposes such as problem solving or debugging. Will use ~\cite{x86:2014} and ~\cite{Oracle:2013} as background into x86 assembly language and Java bytecode. I also have ~\cite{NeuralNetworks:2013} and ~\cite{HeuristicLab:2013} as more research into using FINCH as an EC tool. However since these articles are so short and brief I don't think I will use them as sources in my final paper. 

% The following two commands are all you need in the
% initial runs of your .tex file to
% produce the bibliography for the citations in your paper.
\bibliographystyle{abbrv}
\bibliography{EricBibliography}  % ElenaSample.bib is the name of the Bibliography in this case
% You must have a proper ".bib" file
%  and remember to run:
% pdflatex bibtex pdflatex pdflatex 

\end{document}



